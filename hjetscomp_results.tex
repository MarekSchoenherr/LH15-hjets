\section{RESULTS}
\label{sec:hjetscomp:results}

In the following comparisons, we will typically show the the central
values of each prediction on the left (both as absolute predictions
and in ratio to a reference prediction), and a similar comparison of
the predictions with uncertainty bands on the right. The reason for
the former is that with the overlapping uncertainty bands, it can be
difficult to discriminate the behavior of the central predictions. But
it is also useful to compare the uncertainty bands from each
prediction given similar prescriptions for scale variation.  Note that
it is not enough to say that different predictions agree within their
scale uncertainty bands. In most cases, the predictions should be held
to a higher standard, as the scale logs are common to all of the
calculations that are being compared.

NOTE: I have labelled the version of the plots with uncertainties with
an \_u; they have the same names on the two websites, so the
predictions with the uncertainty bands will have to be explicitly
renamed

NOTE: have to come up with uniform shorthand name for each of the
calculations, i.e. Madgraph5\_aMC@NLO, etc

NOTE: some words as to why the reference distributions have been chosen?

NOTE: haven't said much about the HEJ or the nNLO predictions

\subsection{Inclusive observables}
\label{sec:hjetscomp:results:inclobs}

\begin{figure}[t!]
  \centering
  \includegraphics[width=0.47\textwidth]{figures/hjetscomp_H_y.pdf}
  \caption{
    The inclusive Higgs boson rapidity.
    \label{fig:higgscomp:results:inclobs:hy}
  }
\end{figure}

\begin{figure}[t!]
  \centering
  \includegraphics[width=0.47\textwidth]{figures/hjetscomp_H_pT_incl.pdf}
  \quad
  \includegraphics[width=0.47\textwidth]{figures/hjetscomp_H_pT_excl.pdf}
  \caption{
    The Higgs boson transverse momentum in the inclusive event selection 
    (left) and in the absence of any jet (right).
    \label{fig:higgscomp:results:inclobs:hpt}
  }
\end{figure}

\begin{figure}[t!]
  \centering
  \includegraphics[width=0.47\textwidth]{figures/hjetscomp_NJet_incl_30.pdf}
  \quad
  \includegraphics[width=0.47\textwidth]{figures/hjetscomp_NJet_excl_30.pdf}
  \caption{
    The inclusive (left) and exclusive (right) jet multiplicities.
    \label{fig:higgscomp:results:inclobs:njets}
  }
\end{figure}

\subsection{One-jet observables}
\label{sec:hjetscomp:results:1jobs}

\begin{figure}[t!]
  \centering
  \includegraphics[width=0.47\textwidth]{figures/hjetscomp_H_j_pT_incl.pdf}
  \quad
  \includegraphics[width=0.47\textwidth]{figures/hjetscomp_HT_jets.pdf}
  \caption{
    The Higgs boson transverse momentum in the presence of at least one 
    jet (left) and in the scalar sum of all jet transverse momenta (right).
    \label{fig:higgscomp:results:1obs:hpt_ht}
  }
\end{figure}

\begin{figure}[t!]
  \centering
  \includegraphics[width=0.47\textwidth]{figures/hjetscomp_jet1_pT_incl.pdf}
  \quad
  \includegraphics[width=0.47\textwidth]{figures/hjetscomp_jet1_y.pdf}
  \caption{
    The Higgs boson transverse momentum in the presence of at least one 
    jet (left) and in the scalar sum of all jet transverse momenta (right).
    \label{fig:higgscomp:results:1obs:j1pt_j1y}
  }
\end{figure}

\begin{figure}[t!]
  \centering
  \includegraphics[width=0.47\textwidth]{figures/hjetscomp_Hj_pT_incl.pdf}
  \quad
  \includegraphics[width=0.47\textwidth]{figures/hjetscomp_Hj_pT_excl.pdf}
  \caption{
    The transverse momentum of the Higgs-boson-leading-jet system in the 
    presence of at least one jet (left) and exactly one jet(right).
    \label{fig:higgscomp:results:1obs:hjpt}
  }
\end{figure}


\subsection{Dijet observables}
\label{sec:hjetscomp:results:2jobs}

\begin{figure}[t!]
  \centering
  \includegraphics[width=0.47\textwidth]{figures/hjetscomp_H_jj_pT_incl.pdf}
  \quad
  \includegraphics[width=0.47\textwidth]{figures/hjetscomp_jet2_pT_incl.pdf}
  \caption{
    The transverse momentum of the Higgs boson in the 
    presence of at least two jets (left) and the invariant mass of the 
    Higgs boson and the leading dijet system (right).
    \label{fig:higgscomp:results:2obs:hpt_j2pt}
  }
\end{figure}

\begin{figure}[t!]
  \centering
  \includegraphics[width=0.47\textwidth]{figures/hjetscomp_dijet_mass.pdf}
  \quad
  \includegraphics[width=0.47\textwidth]{figures/hjetscomp_deltay_jj.pdf}
  \caption{
    The transverse momentum of the Higgs boson in the 
    presence of at least two jets (left) and the invariant mass of the 
    Higgs boson and the leading dijet system (right).
    \label{fig:higgscomp:results:2obs:mjj_dyjj}
  }
\end{figure}

\subsection{VBF observables}
\label{sec:hjetscomp:results:VBFobs}

\begin{figure}[t!]
  \centering
  \includegraphics[width=0.47\textwidth]{figures/hjetscomp_deltaphi_jj_VBF.pdf}
  \quad
  \includegraphics[width=0.47\textwidth]{figures/hjetscomp_deltaphi2_VBF.pdf}
  \caption{
    Azimuthal separation of the leading jet pair (left) and 
    $\Delta\phi_2$ (right) after applying VBF cuts.
    \label{fig:higgscomp:results:VBFobs:dphijj_phi2}
  }
\end{figure}




\subsection{Multijet observables}
\label{sec:hjetscomp:results:mjobs}

\begin{figure}[t!]
  \centering
  \includegraphics[width=0.47\textwidth]{figures/hjetscomp_H_jjj_pT_incl.pdf}
  \quad
  \includegraphics[width=0.47\textwidth]{figures/hjetscomp_jet3_pT_incl.pdf}
  \caption{
    The Higgs boson transverse momentum in the presence of at least three 
    jets (left) and transverse momentum of the third jet (right).
    \label{fig:higgscomp:results:mobs:hpt_j3pt}
  }
\end{figure}

\subsection{Jet veto observables}
\label{sec:hjetscomp:results:jvobs}

\begin{figure}[t!]
  \centering
  \includegraphics[width=0.47\textwidth]{figures/hjetscomp_xs_jet_veto_j0.pdf}
  \caption{
    Exclusive zero jet cross section in dependence on the vetoed minimal 
    leading jet transverse momentum.
    \label{fig:higgscomp:results:jvobs:jvxs0}
  }
\end{figure}


\begin{figure}[t!]
  \centering
  \includegraphics[width=0.47\textwidth]{figures/hjetscomp_xs_jet_veto_j1_100.pdf}
  \quad
  \includegraphics[width=0.47\textwidth]{figures/hjetscomp_xs_jet_veto_h_100.pdf}
  \caption{
    Cross section of events with a Higgs boson (left) or a leading jet (right)
    with a transverse momentum of at least 100 GeV in dependence on the 
    vetoed minimal subleading jet transverse momentum.
    \label{fig:higgscomp:results:1obs:jvxs1h_jvxs1j}
  }
\end{figure}

\begin{figure}[t!]
  \centering
  \includegraphics[width=0.47\textwidth]{figures/hjetscomp_xs_central_jet_veto_VBF.pdf}
  \caption{
    Cross section after VBF cuts in dependence with the veto on the leading 
    central jet with transverse momentum larger than 30 GeV in dependence 
    of the rapidity distance of the tagging jets.
    \label{fig:higgscomp:results:1obs:cjvxsvbf}
  }
\end{figure}

