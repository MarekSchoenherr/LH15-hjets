\section{INTRODUCTION}
\label{sec:hjetscomp:intro}

As discussed in \textbf{Section X}, there has been a great deal of progress in
the attack on the Les Houches precision wish list. Such higher order
calculations are needed for the full exploitation of precision LHC
measurements. It is more common in ATLAS and CMS, however, to compare
to predictions involving parton shower Monte Carlos, often
supplemented with matrix element information at leading order (LO) and
next-to-leading order (NLO). Such frameworks allow for the generation
of fully exclusive final states, often more amenable to comparisons
with experimental data. There are a number of such matrix-element plus
parton-shower (ME+PS) frameworks used by the LHC collider experiments,
as can be seen from the predictions in this section. But the higher
(fixed) order calculations often provide the highest precision. It is
thus important to understand: (1) the degree to which the various
ME+PS predictions agree with each other, (2) how well the ME+PS
predictions agree with fixed-order predictions and (3) the impact of
Sudakov regions\,\footnote{By Sudakov region, we refer to kinematic
situations where there is a severe restriction on phase space for
gluon emission, such as the Higgs boson transverse momentum
distribution at low $p_\perp$.} and/or the imposition of jet
vetoes/binning on both fixed-order and ME+PS predictions.  We come to
these comparisons with several expectations: outside of Sudakov regions, the
influence of parton showering/resummation should be mild, and cross
sections that are fairly inclusive (such as the lead-jet transverse
momentum distribution for $h$~+~$\ge1$ jet) should not be subject to
large jet veto logs.

The production of a Higgs boson through gluon--gluon fusion is an
excellent testing ground for such comparisons, both because of the
importance of the process, and the enhanced rate for additional jet
production associated with gluon--gluon fusion. This process of
comparison was started in Les Houches
2013~\cite{AlcarazMaestre:2012vp}, with comparisons of various
predictions for $h$~+~$\ge2$ jets from gluon--gluon fusion, a critical
background for the measurement of vector boson fusion. Here, in this
contribution to the Les Houches 2015 proceedings, we extend that study
to more observables, for a variety of jet multiplicities, and with
comparisons to fixed-order predictions as well as to ME+PS frameworks.

To allow for as standardized a comparison as possible, a group of
generation parameters were agreed upon. MMHT2014 NLO PDFs (and the
NNLO version for the NNLO calculations) were to be used, with a
central value of $\alpha_s(m_Z)=0.118$, and variations thereof of
$\pm0.0015$.  Variations of scale choice are allowed; however, they
should reproduce a scale of $\tfrac{1}{2}m_h$ in the zero-jet limit.

Alas, in some cases, there are deviations from these standards. These
will be noted where present, and the impact on the comparisons will be
discussed.  The Higgs boson was left undecayed. Jets were reconstructed 
with the anti-$k_T$ jet clustering algorithm \cite{Cacciari:2008gp} using
$R=0.4$, and a transverse momentum constraint of $30\gev$ was imposed,
along with a rapidity cut of $\eta(j)<|4.4|$.  To provide a common
framework for the display of the results, a Rivet
routine~\cite{Buckley:2010ar,webpage} was created and distributed to each group
providing a prediction.

In this contribution, predictions have been made with fixed-order
calculations at NLO (using \GoSam~\cite{gosam}), at full
NNLO (using results of the BFGLP group~\cite{Petriello}) \Todo{mention
  \Sherpa NNLO?} and approximate NNLO (using the \Loopsim
approach~\cite{LoopSim} operating on NLO results provided by
\GoSam+\Sherpa~\cite{LoopSim}), and are compared to resummation
calculations~\cite{HqT,HEJ,etc}, to parton shower predictions of
inclusive Higgs boson production at NNLO~\cite{PowhegNNLOPS, SherpaNNLOPS},
and ME+PS predictions~\cite{MG5,Sherpa,Herwig7}, using NLO matrix
element information for up to two (three) jets (\Sherpa). In addition,
fixed-order predictions based on the \GoSam Ntuples using the \Minlo
scales setting procedure (and Sudakov suppression) are presented for
some of the one- and two-jet observables. Technical information for
each of the predictions is provided in the next section.
