\subsection{INTRODUCTION}
\label{sec:hjetscomp:intro}

There has been a great deal of progress in
the attack on the Les Houches precision wish list. Such higher order
calculations are needed for the full exploitation of precision LHC
measurements. However, measurements by the ATLAS and CMS experiments are 
often compared 
to predictions involving parton shower Monte Carlos, often
supplemented with matrix element information at leading order (LO) and
next-to-leading order (NLO). Such frameworks allow for the generation
of fully exclusive final states, often more amenable to comparisons
with experimental data. There are a number of such matrix-element plus
parton-shower (ME+PS) frameworks used by the LHC collider experiments,
as can be seen from the predictions in this section. But the higher
(fixed) order calculations often provide the highest precision. It is
thus important to understand: (1) the degree to which the various
ME+PS predictions agree with each other, (2) how well the ME+PS
predictions agree with fixed-order predictions and (3) the impact of
Sudakov regions\,\footnote{By Sudakov region, we refer to kinematic
situations where there is a severe restriction on phase space for
gluon emission, such as the Higgs boson transverse momentum
distribution at low $p_\perp$.} and/or the imposition of jet
vetoes/binning on both fixed-order and ME+PS predictions.  We come to
these comparisons with several expectations: outside of Sudakov regions, the
influence of parton showering/resummation should be mild, and cross
sections that are fairly inclusive should not be subject to
large jet veto logs. This means that for observables like the leading 
jet's transverse momentum distribution for $h$~+~$\ge1$ jet or the 
inclusive $n$-jet cross sections we do not expect any 
significant resummation correction, originating in either the parton 
showers or the dedicated resummed expressions. On the other hand, 
the more exclusive the cross section, the more different scales are 
involved, the larger the impact of such corrections should be, 
stabilizing the result. Besides the jet vetoed cross sections, 
exclusive $p_\perp$ spectra, both of the Higgs and any jet, should 
be highly dependent on the type of resummation included.

The production of a Higgs boson through gluon--gluon fusion is an
excellent testing ground for such comparisons, because of the
importance of the process, and the enhanced rate for additional jet
production associated with gluon--gluon fusion. First comparisons of 
Higgs production in gluon fusion were done in Les Houches
2013~\cite{AlcarazMaestre:2012vp}, with comparisons of various
predictions for $h$~+~$\ge2$ jets from gluon--gluon fusion, a critical
background for the measurement of vector boson fusion. In the present 
contribution to the Les Houches 2015 proceedings, we extend that study
to more observables, for a variety of jet multiplicities, and with
comparisons to fixed-order predictions as well as to ME+PS frameworks.

To allow for as standardized a comparison as possible, a group of
generation parameters were agreed upon. MMHT2014 NLO PDFs (and the
NNLO version for the NNLO calculations) were to be used, with a
central value of $\alpha_\mathrm{s}(m_Z)=0.118$.  
Variations of scale choice are allowed; however, they
should reproduce a scale of $\tfrac{1}{2}m_h$ in the zero-jet limit. 
All computations were done in the Higgs effective field theory approach 
in the strict $m_\text{top}\to\infty$ limit. Although this does not constitute 
a best-of setup for most contributions these common parameters do not 
alter their underlying methods' properties, capabilites and limits.

Alas, in some cases, there are deviations from these standards. These
will be noted where present, and the impact on the comparisons will be
discussed.  The Higgs boson was left undecayed. Jets were reconstructed 
with the anti-$k_T$ jet clustering algorithm \cite{Cacciari:2008gp} using
$R=0.4$, and a transverse momentum constraint of $30\gev$ was imposed,
along with a rapidity cut of $|\eta(j)|<4.4$.  To provide a common
framework for the display of the results, a Rivet
routine~\cite{Buckley:2010ar,webpage} was created and distributed to each group
providing a prediction.

In this contribution, predictions have been made with fixed-order
calculations at NLO for $pp\to h+1,2,3j$ with standard scale choices 
and the \Minlo approach including Sudakov factors (using \GoSam in 
comnbination with \Sherpa, cf.\ Secs.\ 
\ref{sec:hjetscomp:tools:fo:hnj}-\ref{sec:hjetscomp:tools:fo:hnjminlo}), 
at approximate NNLO (using the \Loopsim approach, cf. Sec.\ 
\ref{sec:hjetscomp:tools:fo:hnjloopsim}) and full NNLO for $pp\to h$ 
(using \Sherpa, cf.\ Sec.\ \ref{sec:hjetscomp:tools:fo:sherpa}) and 
$pp\to h+j$ (using the results of the BFGLP group, cf.\ 
\ref{sec:hjetscomp:tools:fo:BFGLP}). These are compared to explicit 
high-precision resummation calculations for observables of interest, 
i.e.\ the inclusive Higgs boson transverse momentum (using \HqT and \Resbos, cf.\ 
Secs.\ \ref{sec:hjetscomp:tools:ares:hqt} and 
\ref{sec:hjetscomp:tools:ares:resbos}), and the leading-jet $p_\perp$ spectrum 
and jet-vetoed zero jet 
cross section (using the results of the STWZ group, cf.\ Sec.\ 
\ref{sec:hjetscomp:tools:ares:jvres}), as well as to the generic 
parton shower matched predictions of inclusive Higgs boson production 
at NNLO (using \Powheg and \Sherpa, cf.\ Secs.\ 
\ref{sec:hjetscomp:tools:nnlops:powheg} and 
\ref{sec:hjetscomp:tools:nnlops:sherpa}) and the multijet merged 
predictions (provided by \MGaMC, \Herwig and \Sherpa, cf.\ Secs.\ 
\ref{sec:hjetscomp:tools:mc:mgamc}-\ref{sec:hjetscomp:tools:mc:sherpa}), 
using NLO matrix element information for up to two (three) jets (\Sherpa). 
For observables requiring the presence of at least two jets, results 
obtained resumming BFKL-type logarithms (using \Hej, cf.\ Sec.\ 
\ref{sec:hjetscomp:tools:bfkl:hej}) are added. 
Sec.\ \ref{sec:hjetscomp:results} then presents the results in detail 
for a multitude of relevant observables.
