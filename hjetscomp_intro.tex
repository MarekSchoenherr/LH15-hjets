\section{INTRODUCTION}
\label{sec:hjetscomp:intro}

Here is what we did two years ago \cite{AlcarazMaestre:2012vp}.

As discussed in Section X, there has been a great deal of progress in the attack on the Les Houches precision wish list. Such higher order calculations are needed for the full 
exploitation of precision LHC measurements. It is more common in ATLAS and CMS, however, 
to compare to predictions involving parton shower Monte Carlos, often supplemented with matrix element information at leading order and next-to-leading order. Such frameworks 
allow for the generation of fully exclusive final states, often more amenable to 
comparison with experimental data. There are a number of such ME+PS frameworks used by the LHC collider experiments~\cite{MEPS}. But the higher (fixed) order calculations often 
provide the highest precision. It is thus important to understand: (1) the degree to 
which the various ME+PS predictions agree with each other, (2) how well the ME+PS 
predictions agree with fixed order predictions and (3), the impact of Sudakov regions
~\footnote{By Sudakov region, we refer to kinematic situations where there is a
severe restriction on phase space for gluon emission, such as the Higgs boson transverse
momentum distribution at low $p_T$.} 
and/or the imposition of jet vetoes/binning on both fixed order and ME+PS predictions. 
We come to these comparisons with expectations: outside of Sudakov regions, the influence
of parton showering/resummation should be mild, and cross sections that are fairly 
inclusive (such as the lead jet transverse momentum distribution for $H+\ge1$ jet) 
should not be subject to large jet veto logs. 

The production of a Higgs boson through gluon-gluon fusion is an excellent testing ground
for such comparisons, both because of the importance of the process, and the enhanced 
rate for additional jet production associated with gluon-gluon fusion. This process  was started in Les Houches 2015, with comparisons of predictions for $H+\ge2$ jets, critical
for the measurement of vector boson fusion~\cite{2013}. Here, in this contribution to 
the Les Houches 2015 proceedings, we extend that study to more observables, for various 
jet multiplicities. 

To allow for as standardized a comparison as possible, a group of generation parameters 
were agreed upon. MMHT2014 NLO PDFs were to be used, with a central value of 
$\alpha_s(m_Z)$ of 0.118, and variations thereof of $\pm 0.0015$. 

\Todo{do the NNLO predictions used MMHT2014 NNLO?}

Variations of scale choice are allowed; however, they should reproduce a scale of $mH/2$ in the zero jet 
limit. 

\Todo{define general form of scale (for Sherpa?)}

Alas, in some cases, there are deviations from these standards. These will be 
noted where present, and the impact on the comparisons will be discussed. 
The Higgs boson was left undecided. Jets were reconstructed with the antikT4 jet 
clustering algorithm, and a transverse momentum of 30 GeV/c was imposed, along with 
a rapidity cut of $|4.5|$. 
To provide a common framework for the display of the results, a Rivet routine ~\cite{Rivet_routine} was created and distributed to each group providing a prediction. 

In this contribution, predictions have been made with fixed order calculations at NLO 
(gosam~\cite{gosam}), at full NNLO ~\cite{Petriello} and approximate NNLO~\cite{LoopSim}, 
to resummation calculations~\cite{HqT, HEJ, etc}, to parton shower predictions of
inclusive Higgs production at NNLO~\cite{PowhegNNLOPS, SherpaNNLOPS}, and ME+PS
predictions~\cite{MG5,Sherpa,Herwig7}, using NLO matrix element information for 1 and 2
jets. In addition, fixed order predictions with the gosam ntuples using MINLO scales (and Sudakov suppression)  are presented for 
some of the 1 and 2 jet observables.  Technical information is provided in the next section (correct?). 
