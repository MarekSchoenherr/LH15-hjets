\subsubsection{\Herwig}
\label{sec:hjetscomp:tools:mc:herwig}

We provide predictions from NLO merging of $h+0,1,2$ jets at NLO and $h+3,4$
jets at LO in the Higgs effective theory. The merging is carried out with the
\Herwig \ \cite{Bellm:2015jjp} dipole shower module based on
\cite{Platzer:2009jq,Platzer:2011bc}, in a modified version of the algorithms
set out in \cite{Platzer:2012bs,Lonnblad:2012ix}, and implemented in
\cite{Bellm:thesis,Bellm:2016xxx}. The merging implementation will become
publicly available in \Herwig. We use \MGaMC
\cite{Alwall:2014hca} generated amplitudes together with \textsf{ColorFull}
\cite{Sjodahl:2014opa} for the point-by-point evaluation of tree-level type
objects (tree level matrix elements squared, colour- and spin-correlated
matrix elements), and \textsf{OpenLoops} \cite{Cascioli:2011va} for the
evaluation of Born/one-loop interferences.  Subtraction terms and their
integrated counter-parts, phase space generation, integration and process
bookkeeping is handled by the \textsf{Matchbox} module as outlined in
\cite{Bellm:2015jjp}.

The algorithm we use is a modified, unitarized merging algorithm. We allow
finite, higher-order cross section corrections in higher multiplicity jet
bins, but still choose a unitarization procedure to remove potentially
dangerous, logarithmic enhanced terms in inclusive quantities. Below the
merging scale of $30\ {\rm GeV}$, NLO accuracy of the first additional
emission off each contribution is reached by the standard subtractive
matching. Scales are determined through clustering and the core scale 
is defined as $\mu_R=\mu_F=\tfrac{1}{2}\,m_h$. The shower starting scale 
is set to the same value. The uncertainty band is obtained by variation of 
the renormalization and factorization scales of the hard input processes, 
and is covering all other uncertainties present in the algorithm 
(specifically, merging and shower scale variations).
The \texttt{MMHT2014nlo68clas0118} PDF set \cite{Harland-Lang:2014zoa} 
is used.
