\subsubsection{\hjetscompHej}
\label{sec:hjetscomp:tools:bfkl:hej}

High Energy Jets (\hjetscompHej) describes hard, wide angle (high energy-)
emissions to all orders and to all multiplicities. The predictions are based
on events generated according to an all-order resummation, merged with
high-multiplicity full tree-level matrix-elements. The explicit all-order summation
is built on an approximation to the n-parton hard scattering matrix element
\cite{Andersen:2009nu,Andersen:2009he,Andersen:2011hs} which becomes exact in
the limit of wide-angle emissions, ensuring leading logarithmic accuracy for
both real and virtual corrections. A first set of sub-leading logarithmic
terms are included by allowing one un-ordered gluon emission from quarks. All
of these logarithmic terms are important when
the partonic invariant mass is large compared to the typical transverse
momentum in the event. This is precisely the situation which arises in
typical \lq\lq VBF\rq\rq\ cuts, including those used in this study.  Matching to the
full tree level accuracy for up to three jets is obtained by supplementing
the resummation with a merging
procedure~\cite{Andersen:2008ue,Andersen:2008gc}.

The implementation of this framework in a fully-flexible Monte Carlo event
generator is available at \texttt{http://hej.web.cern.ch}, and produces
exclusive samples for events with at least two jets.  The predictions include
resummation also for events with up to two un-ordered emissions,
i.e.~contributions from the first sub-leading configurations.

The factorisation and renormalisation scales can be chosen arbitrarily, just
as in a standard fixed-order calculation. Here, we have chosen to evaluate
two powers of the strong coupling at a scale given by the Higgs mass, and for
the central predictions the remaining scales are evaluated at $\mu_R=\tfrac{1}{2}H_T$. Thus,
for the $n$-jet tree-level evaluation, 
\begin{equation}
  \alpha_\mathrm{s}^{n+2}=\alpha_\mathrm{s}^2(m_h)\cdot \alpha_\mathrm{s}^n(\mu_R).
\end{equation}
The scale variation bands shown in the plots here correspond to varying
$\mu_F,\mu_R\in \{\tfrac{1}{2}\mu_c,\,\tfrac{1}{\sqrt{2}}\mu_c,\,\mu_c,$ $\,\sqrt{2}\mu_c,\,2\mu_c\}$ with 
$\mu_c=\tfrac{1}{2}H_T$, but discarding evaluations where any ratio $\mu_F/\mu_R$ or 
$\mu_R/\mu_F$ is bigger than two (which results in a total of 18 variations
around the central scale). The \texttt{CT10nlo} \cite{Lai:2010vv,Gao:2013xoa} parton 
distribution functions were used in the predictions.
