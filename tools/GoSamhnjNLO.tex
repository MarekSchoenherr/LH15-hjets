\subsubsection{NLO calculation of $pp\to h+1,2,3\,\text{jets}$}
\label{sec:hjetscomp:tools:fo:hnj}

We compute $h+1$ jet, $h+2$ jets and
$h+3$ jets at NLO in QCD in the infinite top mass limit
using \textsc{Sherpa}~\cite{Gleisberg:2008ta} and
\textsc{GoSam}~\cite{Cullen:2011ac,Cullen:2014yla}, linked via the
interface defined in the Binoth Les Houches
Accord~\cite{Binoth:2010xt,Alioli:2013nda}.
The one-loop amplitudes are generated with \textsc{GoSam} employing
\textsc{QGraf}~\cite{Nogueira:1991ex},
\textsc{Form}~\cite{Vermaseren:2000nd,Kuipers:2012rf} and
\textsc{Spinney}~\cite{Cullen:2010jv}, The reduction of the loop
integrals is performed using
\textsc{Ninja}~\cite{Mastrolia:2012bu,vanDeurzen:2013saa,Peraro:2014cba},
\textsc{Golem95}~\cite{Heinrich:2010ax,Binoth:2008uq,Cullen:2011kv}
and \textsc{OneLoop}~\cite{vanHameren:2010cp} for the evaluation of
the scalar integrals.
The calculation of tree-level matrix elements for the Born and the
real emission contribution as well as the subtraction terms in the
Catani-Seymour approach~\cite{Catani:1996vz} have been done within
\textsc{Sherpa} using the matrix element generator
\textsc{Comix}~\cite{Gleisberg:2008fv}.

The computation is performed for a Higgs boson with mass
$m_h=125$, which is considered to be stable. We used the 
\texttt{MMHT2014nlo68clas118} PDF set. We present results 
obtained by processing events pre-generated and
stored in form of \textsc{Root} Ntuples as described
in~\cite{Bern:2013zja}. Theoretical
uncertainties are estimated by varying renormalization and
factorization scales by factors of $\tfrac{1}{2}$ and $2$ 
around the central scale
\begin{equation}
  \mu_0 = \tfrac{1}{2}\,\sqrt{m_{h}^2+\sum p_{T,j_i}^2}\;,
\end{equation}
where $i$ runs over all identified jets.
This scale was chosen to facilitate comparison with the $h+1$ jet NNLO 
calculation of Sec.\ \ref{sec:hjetscomp:tools:fo:BFGLP}.

\Todo{did we really use the scale choice in above equation for GoSam+Sherpa? I thought it was $H'_T/2$ as usual? Yes, we did for the stated reasons.}


\subsubsection{\textsc{MiNLO} calculation of $pp\to h+1,2,3\,\text{jets}$}
\label{sec:hjetscomp:tools:fo:hnjminlo}

Reprocessing the \textsc{Root} Ntuples of Sec.\ 
\ref{sec:hjetscomp:tools:fo:hnj}, for the first time in these 
proceedings we present fixed order NLO results evaluated with a
\textsc{MiNLO}~\cite{Hamilton:2012np} scale chioce, as implemented in
\textsc{Sherpa}. Events are read-in by \textsc{Sherpa}, which applies
the \textsc{MiNLO} prescription event by event. As a \textsc{MiNLO}
core scale we choose
\begin{equation}
  \mu_\text{core}^\text{\textsc{MiNLO}}
  \;=\;\tfrac{1}{2}\,\hat{H}^\prime_T
  \;=\;\tfrac{1}{2}\left(\sqrt{m_h^{2}+p_{T,h}^{2}}
       +\sum_{i}p_{T,i}^{}\right)\,
\end{equation}
where $i$ runs over all partons of the identified core.

\subsubsection{\textsc{LoopSim} merged $\bar{n}$NLO calculation of $pp\to h+(1,2)\,\text{jets}$ and $pp\to h+(2,3)\,\text{jets}$}
\label{sec:hjetscomp:tools:fo:hnjloopsim}

The fixed order Ntuples used for the can be combined using the LoopSim procedure to make
an approximate NNLO prediciton which is missing the double virtual corrections but captures
much of the double unresolved radiation contributions. There is a cut-off dependence on the
additional real radiation since the fixed order Ntuples where generated with a jet $p_T>25$ GeV.

The LoopSim procedure uses a flavour sensitive $k_T$ algoritham where a jet radius of $R=1$ was used.
All other parameters are the same as in the fixed order analysis.
