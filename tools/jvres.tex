\subsubsection{Jet veto resummation}
\label{sec:hjetscomp:tools:ares:jvres}

To resum the $p_\perp$ spectrum of the leading jet as well as the exclusive $0$-jet
(jet-vetoed) cross section, the STWZ predictions~\cite{Stewart:2013faa} utilize
the soft-collinear effective theory (SCET) formalism for jet-veto resummation at
hadron colliders as developed in \cite{Stewart:2009yx, Berger:2010xi,
Tackmann:2012bt, Stewart:2013faa}. The results are obtained in the HEFT 
limit, taking $m_\text{top}\to\infty$.
Jets are defined with a jet radius of $R = 0.4$ and without any cut on the jet
rapidity. The \texttt{MMHT2014nnlo68cl} PDFs are used. The calculation is carried out to
NNLL$'+$NNLO order. The resummation is performed by renormalization group
evolution in virtuality and rapidity space in SCET. The NNLL$'$ resummation
includes the RG evolution at next-to-next-to-leading logarithmic order together
with the full NNLO singular matching correction, thus incorporating all two-loop
virtual and singular real-emission contributions in the resummation. This allows
to perform the matching to the full NNLO result by adding purely nonsingular
corrections to the resummed contributions, and to achieve a smooth transition to
the fixed-order result simply by turning off the RG evolution using profile
scales~\cite{Ligeti:2008ac, Abbate:2010xh}. The perturbative uncertainties are
estimated by varying the relevant virtuality and rapidity renormalization scales
using profile scale variations, which has been established as a reliable method
to assess perturbative uncertainties in resummed predictions. We evaluate
separate fixed-order and resummation uncertainties which are added in
quadrature~\cite{Berger:2010xi, Stewart:2011cf, Stewart:2013faa}.
The predictions use a complex value for the hard scale $\mu_H = -i \mu_{FO}$ where
$\mu_{FO} = m_h$ is the fixed-order scale, which allows to resum large virtual
corrections in the $gg\to h$ form factor in both the 0-jet limit and the
inclusive cross section. This scale choice results in a similar 
inclusive cross section compared to a standard NNLO calculation with 
$\mu_R=\mu_F=\tfrac{1}{2}m_h$. The uncertainty related to this resummation is
estimated by varying the phase of $\mu_H$ and is also added in quadrature.

