\subsubsection{\HqT}
\label{sec:hjetscomp:tools:ares:hqt}

\HqT \cite{Bozzi:2005wk,deFlorian:2011xf} is a numerical program which 
combines the exact fixed order calculation of the transverse momentum 
spectrum valid at large $q_T$ at ${\cal O}(\alpha_\mathrm{s}^4)$ with the 
resummation of the logarithmically enhanced contributions at small 
transverse momenta at next-to-next-to-leading logarithmic accuracy.
The calculation is performed according to the formalism of 
\cite{Catani:2000vq,Bozzi:2005wk} and it implements a unitarity 
constraint such that, upon integration over $q_T$, the inclusive 
NNLO cross section is recovered. The results are valid in the large 
$m_{top}$ approximation. As any perturbative QCD computation in hadron 
collisions, the results depend on the factorization ($\mu_F$) and 
renormalization ($\mu_R$) scales. In addition, the resummation procedure 
introduces an additional scale, dubbed \lq\lq resummation scale\rq\rq\ 
($Q$). The three scales must be chosen of the order of the hard scale 
of the process, $m_h$. The numerical results presented here are obtained 
by using \HqT-2.0 with $\mu_F=\mu_R=Q=\tfrac{1}{2}\,m_h$ GeV as central scale choice. 
The procedure to estimate perturbative uncertainties is to perform 
independent variations of $\mu_F$, $\mu_R$ and $Q$ around the central 
value by a factor of 2 with the constraints $\tfrac{1}{2} < \mu_F/\mu_R < 2$ and 
$\tfrac{1}{2} <Q/\mu_R<2$.