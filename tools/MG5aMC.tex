\subsubsection{\MGaMC}
\label{sec:hjetscomp:tools:mc:mgamc}

Higgs production in gluon fusion, in association with multiple jets,
is generated with \MGaMC \cite{Alwall:2014hca}~at NLO
accuracy using the following commands
\begin{verbatim}
  import model HC_NLO_X0_UFO-heft
  generate p p > x0 /t [QCD] @0
  add process p p > x0 j /t [QCD] @1
  add process p p > x0 j j /t [QCD] @2
  output MG5aMC_FxFx_Hjets
\end{verbatim}
The first command loads the model that includes the Higgs boson
effective coupling to gluons in the $m_t\to\infty$ limit. This model
can be found on the
FeynRules~\cite{Alloul:2013bka}~website\footnote{\texttt{http://feynrules.irmp.ucl.ac.be}}
and was originally created for the studies in
\cite{Demartin:2014fia}. In this model, the Standard Model Higgs
boson is called `\texttt{x0}', and therefore the second to fourth
commands generate this boson in association with 0, 1 and 2 jets,
respectively. The model, and therefore also the definitions of
\texttt{p} and \texttt{j}, are in the 5 flavour scheme. The
`\texttt{/t}' syntax is needed to remove the explicit top quark
contributions from the loops (they are already integrated out in the
effective Higgs-gluon vertices). By setting the parameter
\texttt{ickkw} to \texttt{3} in the \texttt{run\_card.dat}, the FxFx
merging~\cite{Frederix:2012ps}~is turned on. The LHE events are
matched to the \Pythia 8 (v.210) parton
shower~\cite{Sjostrand:2014zea}, using the FxFx interface also used in
\cite{Frederix:2015eii}. As central choices for the factorisation
and renormalisation scales we use the default value in the
\MGaMC code, which, in the context of FxFx merging, is
roughly given by the transverse energy of the Higgs boson, after the
partons entering the matrix elements have been clustered to a $pp \to h
j$ configuration. $pp\to h$ configurations thus are calculated using 
$m_h$ as scale. The central merging scale is taken to be 35 GeV,
while the variations include 25 GeV and 50 GeV. These scales include 
values both below as well as above the default minimal jet transverse
momentum used in the analysis and should therefore cover the complete
range of uncertainties coming from the merging.  The uncertainty band
is computed by varying the factorisation and renormalisation scales by
a factor of 2 up and down from the central value, using the reweighting
technique as described in \cite{Frederix:2011ss}, for each of the
three choices of merging scales. It is therefore obtained
as the bin-by-bin envelope of $3 \times 3 \times 3 = 27$ individual
values. This is similar to what is done in
\cite{Frederix:2015eii}---apart from the uncertainties coming
from the parton distribution function, which are not taking into
account here. Throughout the \texttt{MMHT2014nlo68cl} PDF set with 
$\alpha_s(m_Z)=0.12$ has been used.
