\section{CONCLUSIONS}
\label{sec:hjetscomp:conclusions}


With the advent of the 13 TeV running at the LHC, and with the expected 
data sample, precision Higgs boson measurements will now be possible. The 
largest Higgs boson production process is gluon fusion. A variety of theoretical 
tools exist for predictions for Higgs boson (+jets) production in this 
channel. As higher order corrections are especially sizable for $gg\to h$, 
it is important to understand the accuracies and regions of applicability 
of the various predictions. Too often, the comment has been that since 
the predictions agree within the theoretical uncertainties, all is well. 
However, a higher standard must be used, as the predictions have many of 
the theoretical uncertainties, such as scale uncertainties, in common. 

We have compared fixed order predictions at NLO and NNLO (including 
approximate NNLO), resummed predictions, NNLO predictions matched to parton 
showers, multjet merged predictions at NLO accuracy, and resummations 
in the BFKL limit. This allows a better understanding of (1) how 
consistent are calculations which should be consistent and (2) the impact 
of soft gluon radiation and higher order corrections. All predictions have 
been carried out without non-perturbative corrections to allow for a one-to-one 
comparisons. A few observations follow. 

NNLO effects can change not only the normalization of distributions, but 
also the shape. For example, NLO predictions of the Higgs rapidity 
distribution are all similar in normalization and shape; however, the 
NLO predictions fall off more rapidly at high rapidity than predictions 
at NNLO. The differences between NLO and NNLO, however, are only 
noticeable in regions beyond the kinematic cuts applied at the LHC. It is 
interesting that, for the scale choices used in this study, there is 
neither a shape nor normalization shift from the NNLO corrections for the 
inclusive lead jet $p_\perp$ distribution while they are present for 
the Higgs boson $p_\perp$ in the presence of at least one jet.. 

The highest precision for inclusive jet multiplicity distributions is 
present with fixed order predictions, either at NLO or NNLO; in general, 
predictions from resummed/parton shower programs agree well with the 
fixed order predictions within their expected accuracy. The most 
extensive comparisons in this study are with respect to the $p_\perp$ 
distribution for the lead jet in $h+\ge1$ jet events. With the recent 
NNLO calculation for this quantity, the uncertainties are less than 10\%. 
As mentioned above, the NNLO corrections are small. The \NNLOPS 
predictions for this variable are in good agreement with each other 
as well as with the NLO/NNLO predictions for $p_\perp \le$ 100 GeV/c, with 
some separation between the \NNLOPS results at higher transverse momentum. 
The multijet jet merged predictions agree well with NLO/NNLO at low $p_\perp$, 
but fall below by about 15-20\% at high $p_\perp$. The STWZ prediction 
agrees well at low $p_\perp$, but rises above the fixed order results 
by about 20\% at high $p_\perp$. One of the key take-home points is that 
the introduction of a parton shower or a resummation should not greatly 
affect the fixed order results, for observables that are suitably 
inclusive. These conclusions are largely true for comparisons for the 
sub-leading and third-leading jet as well. The situation is more complex 
for exclusive final states, where a jet veto is applied to any additional 
jet. Here, there are jet veto logs that have to be re-summed. We note, 
though, in general that there is still good agreement among the 
predictions used here. Although, this is not explicitly part of this 
study, we note that the impact of jet veto logarithms (after resummation) 
on the NNLO prediction for $h+\ge1$ jet are small, indicating again that 
the fixed order predictions for that quantity should be 
reliable~\cite{Banfi:2012jm,Banfi:2015pju}.

Resummed/parton shower predictions provide a better description for 
observables where Sudakov effects are important, such as the transverse 
momentum distribution for inclusive (exclusive) Higgs production. For 
the inclusive case, all predictions agree well with each other (and with 
the reference \HqT), with some deviations observed at the lowest $p_\perp$ 
values. Differences are more evident for the exclusive case, where the 
high transverse momentum for the Higgs boson must be supplied by a combination of 
jets lower than the $30\,\gev$ cutoff and soft gluon radiation. Although 
the fixed order predictions are unstable at low $p_\perp$, there is good 
agreement with the resummed/parton shower predictions at high $p_\perp$, 
where the bulk of the transverse momentum for the resummed/parton shower 
prediction is provided by the hard matrix element. 



\section*{ACKNOWLEDGEMENTS}

We thank the organisers.
MS acknowledges support by the Swiss National Science Foundation (SNF) 
under contract PP00P2-128552. 
The work of SH, YL and SP was supported by the U.S. Department of Energy 
under contract DE--AC02--76SF00515.
